\documentclass[structabstract]{article}
\usepackage[varg]{txfonts}
% include packages
%\usepackage[dvips]{graphicx}
\usepackage{url}
\usepackage[breaklinks=true]{hyperref}
\usepackage{twoopt}
\usepackage{natbib}
\bibpunct{(}{)}{;}{a}{}{,}%% natbib format for A&A and ApJ
\usepackage{ctable}
\usepackage{multirow}
\usepackage[farskip=0pt]{subfig}

\setlength{\textwidth}{6.5in}
\setlength{\textheight}{9in}
\setlength{\topmargin}{-0.0625in}
\setlength{\oddsidemargin}{0in}
\setlength{\evensidemargin}{0in}
\setlength{\headheight}{0in}
\setlength{\headsep}{0in}
\setlength{\hoffset}{0in}
\setlength{\voffset}{0in}

\begin{document}

%%%%%%%%%%%%%%%%%%%%%%%%%%%%%%%%%%%%%%%%%%%%%%%%%%%%%%%%%%%%%%%%%%%%%%%%%%%%%%%%%%%%%%%%%%%%%%%%%%CUTCUTCUT

%\svnInfo $Id: bbs.tex 12951 2014-03-08 18:40:31Z dijkema $
\section[Factor: Facet Calibration for LOFAR]{Factor: Facet Calibration for
LOFAR\footnote{This section is maintained by David Rafferty
({\tt drafferty@hs.uni-hamburg.de}).}}
\label{factor}

Factor is a Python package which allows for the production of wide-field HBA
LOFAR images. Note that Factor is still in beta. Please report bugs to
drafferty@hs.uni-hamburg.de. To initialize your environment for Factor, users on
CEP2 and CEP3 should run the following commands:
\begin{verbatim}
use LofIm
source ~rafferty/init_factor
use Casa
use Pythonlibs
\end{verbatim}

%-----------------------------------------------------------
\subsection{Usage}
\label{factor:usage}

Factor can be run from the command-line as follows:
\begin{verbatim}
Usage: factor <parset>
Options:
  --version   show program's version number and exit
  -h, --help  show this help message and exit
  -v          Verbose
\end{verbatim}
The parset specifies the parameters of the run. These are described in the next
sections.

%-----------------------------------------------------------
\subsection{Parset}
\label{factor:parset}

This is an example parset that shows the available parameters:
\begin{verbatim}
[global]
# LOFAR installation root directories. If not given, they will be determined
# from the environment
lofarroot = /lofar/build/gnu_opt/installed
lofarpythonpath = /lofar/build/gnu_opt/installed/lib/python2.7/dist-packages

# Full path to Generic Pipeline root directory
piperoot = /path/to/genericpipe/dir

# Full path to the cluster description file. If not given, the clusterdesc file
# for a single (i.e., local) node is used
clusterdesc = /path/to/clusterdesc

# Full path to working dir - other paths are relative to this
dir_working = /data/wdir

# Full path to directory containing selfcaled bands, will be scanned for all .MS
# and .ms files
dir_ms = /data/bands

# Parmdb name for dir-indep. selfcal solutions (stored inside the input
# band measurement sets, so path should be relative to those)
parmdb_name = instrument_ap_smoothed

# Full path to file containing calibrator directions. If not given, directions
# are selected internally using the flux and size cuts below (min flux, max size
# of a source, and max separation between sources below which they are grouped
# into one direction). The number of internally derived directions can be
# limited to a maximum number of directions if desired.
directions_file = /data/directions.txt
directions_flux_min_Jy = 0.1
directions_size_max_arcmin = 3.0
directions_separation_max_arcmin = 6.0
directions_max_num = 50

# Maximum number of directions to process (default = all)
ndir = 10

# Maximum number of directions to process in parallel (default = 1)
ndir_parallel = 4

# Maximum number of CPUs per node to use (default = all)
ncpu = 6

# Full path to cluster description file. Use clusterdesc_file = PBS to use
# the PBS / torque reserved nodes. Set distribute to True if the cluster
# has a distributed file system. Set to False if the cluster has a shared
# file system (the default)
clusterdesc_file = PBS
distribute = False (default = False)

# Use interactive mode (default = False)
interactive = False

# Use ftw to FFT model image instead of BBS predict (default = True)
use_ftw = True

# Make final mosaic (default = True)
make_mosaic = False

# MS-specific parameters
[ms1.ms]
init_skymodel = /data/ms1strangename.sky
param1 = 123
param2 = 123

[ms2.ms]
init_skymodel = /data/ms2strangename.sky
param1 = 123
param2 = 123
\end{verbatim}

%-----------------------------------------------------------
\subsection{Directions}
\label{factor:directions

The directions for which DDE calibration is performed can be selected automatically
by factor or specified in a file. The format of the file is as follows:

\begin{verbatim}
# This is an example directions file for Factor.
#
# Directions should be sorted in order of reduction (e.g., bright to faint).
#
# Columns are defined as follows (in [] are optional parameters with their default values):
# Name (str), RA (float), DEC (float), [regionfile (str, default='')],
# [multiscale (bool, default=True)], [solint_amp (int, default=600)],
# [solint_ph (int, default=5)], [make_final_image (bool, default=True)],
# [cal_radius (float arcmin, default=3.0)]

calibrator_1, 111.11, 22.22, , , , , ,
calibrator_2, 222.22, 33.33, reg/cal2.reg, True, 600, 5, False, 4.5
calibrator_3, 333.33, 44.44, , True, 600, 5, ,
\end{verbatim}


%-----------------------------------------------------------
\subsection{Parallelization}
\label{factor:parallel}

Factor has been designed to run as much as possible in a parallel manner. To this
end, support is present for distributing the reduction over nodes of a compute
cluster and over the processors of a single computer.

To use Factor on a single computer, no setup is necessary. To use factor on
multiple nodes of a compute cluster, the required setup depends on the cluster
layout (distributed vs. shared file system) and scheduling system (if any).

\subsubsection{Torque / PBS scheduler}
For use on a compute cluster that uses torque and PBS, set 'clusterdesc = PBS'.
Factor will then automatically determine the nodes for which you have a PBS
reservation and use them.

\subsubsection{Shared file system}
Factor works by default on a compute cluster with a shared file system.

\subsubsection{Distributed file system}
For use on a cluster with a distributed file system (such as CEP2 and CEP3), set
'distribute = True'. Factor will then synchronize files between the available
nodes. Note that all data will be mirrored on each node, so all nodes should
have sufficient disk space. Additionally, the local disks must have the same
name on each node.

%-----------------------------------------------------------
\subsection{Example Reduction}
\label{factor:example}

A example reduction is shown below:

\begin{verbatim}
$ factor factor.parset
INFO - parset - Reading parset file: factor.parset
INFO - parset - Working directory is Test_run
INFO - parset - Working on 4 bands
INFO - directions - Reading directions file: factor_directions.txt
INFO - root - Sky models found for all MS files. Skipping initial subtraction operation.
INFO - root - No directions file given. Selecting directions internally...
INFO - LSMTool.Filter - Kept 138 patches.
INFO - directions - Found 138 directions with fluxes above 0.2 Jy
INFO - LSMTool.Filter - Kept 137 patches.
INFO - directions - Found 137 directions with fluxes above 0.2 Jy and sizes below 3.0 arcmin
INFO - directions - Merging directions within 6.0 arcmin of each other...
INFO - LSMTool.Filter - Removed 34 patches.
INFO - directions - Kept 100 directions in total
INFO - directions - Writing directions file: factor_directions.txt
INFO - directions - Reading directions file: factor_directions.txt
INFO - FacetAddCal - <-- Operation FacetAddCal started (direction(s): mask_patch_497)
INFO - FacetAddCal - Selecting sources for this direction...
INFO - FacetAddCal - Adding sources for this direction...
INFO - FacetAddCal - Phase shifting DATA...
INFO - FacetAddCal - --> Operation FacetAddCal finished (direction(s): mask_patch_497)
INFO - FacetSetup - <-- Operation FacetSetup started (direction(s): mask_patch_497)
INFO - FacetSetup - Applying direction-independent calibration...
INFO - FacetSetup - Averaging DATA...
INFO - FacetSetup - Averaging CORRECTED_DATA...
INFO - FacetSetup - Concatenating bands...
INFO - FacetSetup - --> Operation FacetSetup finished (direction(s): mask_patch_497)
INFO - FacetSelfcal - <-- Operation FacetSelfcal started (direction(s): mask_patch_497)
INFO - FacetSelfcal - Imaging (facet image #0)...
INFO - FacetSelfcal - FFTing model image (facet model #0)...
INFO - FacetSelfcal - Solving for phase solutions and applying them (#1)...
INFO - FacetSelfcal - Imaging (facet image #1)...
INFO - FacetSelfcal - FFTing model image (facet model #1)...
INFO - FacetSelfcal - Solving for phase solutions and applying them (#2)...
INFO - FacetSelfcal - Imaging (facet image #2)...
INFO - FacetSelfcal - FFTing model image (facet model #2)...
INFO - FacetSelfcal - Solving for amplitude solutions and applying them (#1)...
INFO - FacetSelfcal - Merging instrument parmdbs...
INFO - FacetSelfcal - Smoothing amplitude solutions...
INFO - FacetSelfcal - Applying amplitude solutions...
INFO - FacetSelfcal - Imaging (facet image #3)...
INFO - FacetSelfcal - FFTing model image (facet model #3)...
INFO - FacetSelfcal - Preapplying amplitude solutions...
INFO - FacetSelfcal - Solving for amplitude solutions (#2)...
INFO - FacetSelfcal - Merging instrument parmdbs...
INFO - FacetSelfcal - Smoothing amplitude solutions...
INFO - FacetSelfcal - Applying amplitude solutions...
INFO - FacetSelfcal - Imaging (facet image #4)...
INFO - FacetSelfcal - Merging final instrument parmdbs...
INFO - FacetSelfcal - Smoothing amplitude solutions...
INFO - FacetSelfcal - --> Operation FacetSelfcal finished (direction(s): mask_patch_497)
INFO - FacetAddAll - <-- Operation FacetAddAll started (direction(s): mask_patch_497)
INFO - FacetAddAll - Selecting sources for this direction...
INFO - FacetAddAll - Adding sources for this direction...
INFO - FacetAddAll - Phase shifting DATA...
INFO - FacetAddAll - --> Operation FacetAddAll finished (direction(s): mask_patch_497)
INFO - FacetImage - <-- Operation FacetImage started (direction(s): mask_patch_497)
INFO - FacetImage - Applying direction-dependent calibration...
INFO - FacetImage - Averaging DATA...
INFO - FacetImage - Concatenating bands...
INFO - FacetImage - Imaging...
\end{verbatim}


\end{document}
